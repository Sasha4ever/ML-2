%%% Поля и разметка страницы %%%
\documentclass[a4paper,12pt]{article}
\usepackage{lscape}		% Для включения альбомных страниц

%%% Кодировки и шрифты %%%
\usepackage{cmap}						% Улучшенный поиск русских слов в полученном pdf-файле
\usepackage[T2A]{fontenc}				% Поддержка русских букв
\usepackage[utf8]{inputenc}				% Кодировка utf8
\usepackage[english, russian]{babel}	% Языки: русский, английский
% \usepackage{pscyr}						% Красивые русские шрифты

%%% Математические пакеты %%%
\usepackage{amsthm,amsfonts,amsmath,amssymb,amscd} % Математические дополнения от AMS

%%% Оформление абзацев %%%
%\usepackage{indentfirst} % Красная строка

%%% Цвета %%%
\usepackage[usenames]{color}
\usepackage{color}
\usepackage{colortbl}

%%% Таблицы %%%
\usepackage{longtable}					% Длинные таблицы
\usepackage{multirow,makecell,array}	% Улучшенное форматирование таблиц

%%% Общее форматирование
\usepackage[singlelinecheck=off,center]{caption}	% Многострочные подписи
\usepackage{soul}									% Поддержка переносоустойчивых подчёркиваний и зачёркиваний

%%% Библиография %%%
\usepackage{cite} % Красивые ссылки на литературу

%%% Гиперссылки %%%
\usepackage[plainpages=false,pdfpagelabels=false]{hyperref}
\definecolor{linkcolor}{rgb}{0.9,0,0}
\definecolor{citecolor}{rgb}{0,0.6,0}
\definecolor{urlcolor}{rgb}{0,0,1}
\hypersetup{
    colorlinks, linkcolor={linkcolor},
    citecolor={citecolor}, urlcolor={urlcolor}
}

%%% Изображения %%%
\usepackage{graphicx}		% Подключаем пакет работы с графикой
\graphicspath{{images/}}	% Пути к изображениям

%%% Выравнивание и переносы %%%
\sloppy					% Избавляемся от переполнений
\clubpenalty=10000		% Запрещаем разрыв страницы после первой строки абзаца
\widowpenalty=10000		% Запрещаем разрыв страницы после последней строки абзаца


\usepackage{tikz}


%%%%%%%%%%%%%%%%%%%%%%%%%%%%%%%%%%%%%%%%%%%%%%%%%%%%%%%%%%%%%%%%%%%%%%%%%%%%%%%%%%%
\begin{document}
\begin{center}
\Huge{Домашнее задание по курсу $"$Математическая логика - 2$"$}
\end{center}

\section{Язык и аксиоматика теории множеств}
\subsection*{$\S$ 1.3}
\paragraph*{Условие}
Доказать, что $\varnothing \not= 	\{ \varnothing \}.$
\paragraph*{Доказательство}
По определению\\
$ x = y \rightleftharpoons \forall t ( t \in x \Leftrightarrow t \in y ).$\\
Пусть $\varnothing = 	\{ \varnothing \} $,
$ \Rightarrow \forall t ( t \in \{ \varnothing \} \Leftrightarrow t \in \varnothing )$
Противоречие для t = $ \varnothing $

\subsection*{$\S$ 1.4}
\paragraph*{Условие}
Доказать, что $\{\{1, 2\}, \{2,3\}\} \not= 	\{ 1,2,3 \}. $
\paragraph*{Доказательство}
По определению\\
$ x = y \rightleftharpoons \forall t ( t \in x \Leftrightarrow t \in y ).$\\
Пусть $\{\{1, 2\}, \{2,3\}\} = 	\{ 1,2,3 \} $,
$ \Rightarrow \forall t ( t \in \{ 1,2,3 \} \Leftrightarrow t \in \{\{1, 2\}, \{2,3\}\} )$
Противоречие для t = $ 1 $

\subsection*{$\S$ 1.6}
\paragraph*{Условие}
Доказать, что $\exists$ лишь одно множество, не имеющее элементов.
\paragraph*{Доказательство}
Пусть $\exists$ два множества X и $X_0$, не имеющих элементов и такие, что $X \not= X_0$\\
$ \Rightarrow \exists t ( t \in X \Rightarrow  t \not\in X_0 )$\\
Противоречие так как $\nexists t \in$ X.

\subsection*{$\S$ 1.8}
\paragraph*{Условие}
Доказать, что множество всех корней многочлена $\alpha (x)=\beta (x) \gamma (x)$ есть объединение множеств корней $\beta (x)$ и $\gamma(x)$.
\paragraph*{Доказательство}
Чтобы докаказать, что множество корней = объединения множеств, надо доказать, что любой корень является либо корнем $\beta (x)$ либо $\gamma(x)$ и что других корней не существует.\\
1) Пусть существует корень $x_0$, который не является корнем ни $\beta (x)$, ни корнем $\gamma(x)$ \\
$\Rightarrow \alpha (x_0) = 0, \beta (x_0) \not= 0,  \gamma (x_0) \not= 0$. Противоречие
2) Пусть $x_0$ корень $\beta (x)$ или  $\gamma(x)$ , тогда  $\beta (x_0) = 0$ или $\gamma(x_0) = 0$ $\Rightarrow \alpha(x_0) = 0$

\subsection*{$\S$ 1.9}
\paragraph*{Условие}
Доказать, что персечение множеств действительных корней многочленов  $\alpha (x) и \beta (x)$ с действительными коэффицентами  совпадает с множеством всех действительных корней $\gamma(x) =\alpha^2 (x) + \beta^2 (x)  $.
\paragraph*{Доказательство}
Чтобы докаказать, что множество корней = персечение множеств, надо доказать, что любой корень из пересейчения является корнем и что других корней не существует.\\
1)Если $x_0$ корень $\alpha (x) и \beta (x)$ $\Rightarrow$ $\gamma(x_0) = 0$
2)Пусть существует корень  $\gamma(x) x_0$, который не является корнем ни $\alpha (x)$, ни корнем $\beta(x)$ \\
Тогда $\gamma(x_0) = 0$ $\Rightarrow \alpha^2 (x_0) + \beta^2 (x_0) = 0 \Rightarrow  \alpha(x_0) =  0 \& \beta (x_0) = 0$

\subsection*{$\S$ 1.11 (а, г, ж)}
\paragraph*{Условие}
Доказать следующие тождества\\
a)$ A \cup A = A \cap A = A $
\paragraph*{Доказательство}
Распишем по определению\\
$\{ Z \mid (Z \in A \vee Z \in A)\} = \{ Z \in A \cup A \mid Z \in A \wedge Z \in A \} = A $\\
Упростим\\
$\{ Z \mid (Z \in A )\} = \{ Z \in A \cup A \mid Z \in A \} = A $ $\Leftrightarrow$ $ A = \{ Z \in A  \mid Z \in A \} = A $  $\Leftrightarrow$ $A=A=A$
\paragraph*{Условие}
г)$ A \cap ( B \cap C ) = ( A \cap B ) \cap C$
\paragraph*{Доказательство} \mbox{}\\
\def\firstcircle{(0,0) circle (1.5cm)}
\def\secondcircle{(45:2cm) circle (1.5cm)}
\def\thirdcircle{(0:2cm) circle (1.5cm)}
\begin{tikzpicture}
	\begin{scope}[shift={(0cm,6cm)}, fill opacity=0.8]
		\begin{scope}% first circle without the second
			\clip \secondcircle;
      		\fill[red] \thirdcircle;
        \end{scope}
        \begin{scope}% first circle without the second
			 \clip \firstcircle;
      		\clip \secondcircle;
     		 \fill[green] \thirdcircle;
        \end{scope}
        \draw \firstcircle node {$A$};
        \draw \secondcircle node {$B$};
        \draw \thirdcircle node {$C$};
    \end{scope}

	 \begin{scope}[shift={(6cm,6cm)}, fill opacity=0.8]
	 	\begin{scope}% first circle without the second
			\clip \firstcircle;
      		\fill[red] \secondcircle;
        \end{scope}
	 	\begin{scope}% first circle without the second
			 \clip \firstcircle;
      		\clip \secondcircle;
     		 \fill[green] \thirdcircle;
        \end{scope}
        \draw \firstcircle node {$A$};
        \draw \secondcircle node  {$B$};
        \draw \thirdcircle node {$C$};
    \end{scope}
\end{tikzpicture}

\paragraph*{Условие}
ж)$A \cup ( B \cap C ) = ( A \cup B ) \cap ( A \cup C )$
\paragraph*{Доказательство} \mbox{}\\

\begin{tikzpicture}
	\begin{scope}[shift={(0cm,6cm)}, fill opacity=0.5]
		\begin{scope}% first circle without the second
			\clip \secondcircle;
      		\fill[red] \thirdcircle;
        \end{scope}
        \begin{scope}% first circle without the second
     		 \fill[green] \firstcircle;
     		 \clip \secondcircle;
      		\fill[green] \thirdcircle;
        \end{scope}
        
        \draw \firstcircle node {$A$};
        \draw \secondcircle node {$B$};
        \draw \thirdcircle node {$C$};
    \end{scope}

	 \begin{scope}[shift={(6cm,6cm)}, fill opacity=0.5]
	 	\begin{scope}% first circle without the second
	 		\fill[red] \firstcircle;
	 		\fill[red] \secondcircle;
        \end{scope}
        \begin{scope}% first circle without the second
			\fill[yellow] \firstcircle;
      		\fill[yellow] \thirdcircle;
        \end{scope}
	 	\begin{scope}% first circle without the second
        \end{scope}
        \draw \firstcircle node {$A$};
        \draw \secondcircle node  {$B$};
        \draw \thirdcircle node {$C$};
    \end{scope}
\end{tikzpicture}

\subsection*{$\S$ 1.12(в, д, ж, п, т)}

\subsection*{$\S$ 1.13(а, д, к)}

\subsection*{$\S$ 1.14(в, к)}

\subsection*{$\S$ 1.15}
\paragraph*{Условие}
Доказать, что\\
a) $(A_1 \cup ... \cup A_n) \bigtriangleup (B_1 \cup ... \cup B_n) \subseteq (A_1 \bigtriangleup B_1) \cup ... \cup (A_n \bigtriangleup B_n) $
\paragraph*{Доказательство}
Докажем по индукции:\\
\textbf{База индукции}\\
 n=1)  $(A_1) \bigtriangleup (B_1) \subseteq (A_1 \bigtriangleup B_1) $ (очевидно)\\
 n=2)  $(A_1 \cup  A_2) \bigtriangleup (B_1 \cup B_2) \subseteq (A_1 \bigtriangleup B_1) \cup (A_2 \bigtriangleup B_2) $(Доказывалось на уроке)\\
 \textbf{Преположение индукции}\\
 Пусть верно для $\forall n < k$\\
 \textbf{Шаг индукции}\\
 Докажем для k+1\\
 $(A_1 \cup ... \cup A_k+1) \bigtriangleup (B_1 \cup ... \cup B_k+1) \subseteq (A_1 \bigtriangleup B_1) \cup ... \cup (A_k+1 \bigtriangleup B_k+1) $\\
 пусть $ A_0 = A_1 \cup ... \cup A_k и B_0 = B_1 \cup ... \cup B_k$\\
 $(A_1 \cup ... \cup A_{k+1}) \bigtriangleup (B_1 \cup ... \cup B_{k+1}) \Leftrightarrow (A_0 \cup A_{k+1}) \bigtriangleup (B_0 \cup B_{k+1}) \subseteq $ \\
$ \subseteq (A_0 \bigtriangleup B_0) \cup (A_k \bigtriangleup B_k)$\\
$(A_0 \bigtriangleup B_0) \subseteq  (A_1 \bigtriangleup B_1) \cup ... \cup (A_k \bigtriangleup B_k)$\\
$\Rightarrow (A_1 \cup ... \cup A_k+1) \bigtriangleup (B_1 \cup ... \cup B_k+1) \subseteq (A_1 \bigtriangleup B_1) \cup ... \cup (A_k+1 \bigtriangleup B_k+1) $

\paragraph*{Условие}
б) $(A_1 \cap ... \cap A_n) \bigtriangleup (B_1 \cap ... \cap B_n) \subseteq (A_1 \bigtriangleup B_1) \cup ... \cup (A_n \bigtriangleup B_n) $
\paragraph*{Доказательство}
Докажем по индукции:\\
\textbf{База индукции}\\
 n=1)  $(A_1) \bigtriangleup (B_1) \subseteq (A_1 \bigtriangleup B_1) $ (очевидно)\\
 n=2)  $(A_1 \cap  A_2) \bigtriangleup (B_1 \cap B_2) \subseteq (A_1 \bigtriangleup B_1) \cup (A_2 \bigtriangleup B_2) $(Доказывалось на уроке)\\
 \textbf{Преположение индукции}\\
 Пусть верно для $\forall n < k$\\
 \textbf{Шаг индукции}\\
 Докажем для k+1\\
 $(A_1 \cap ... \cap A_k+1) \bigtriangleup (B_1 \cap ... \cap B_k+1) \subseteq (A_1 \bigtriangleup B_1) \cup ... \cup (A_k+1 \bigtriangleup B_k+1) $\\
 пусть $ A_0 = A_1 \cap ... \cap A_k и B_0 = B_1 \cap ... \cap B_k$\\
 $(A_1 \cap ... \cap A_{k+1}) \bigtriangleup (B_1 \cap ... \cap B_{k+1}) \Leftrightarrow (A_0 \cap A_{k+1}) \bigtriangleup (B_0 \cap B_{k+1}) \subseteq $ \\
$ \subseteq (A_0 \bigtriangleup B_0) \cup (A_k \bigtriangleup B_k)$\\
$(A_0 \bigtriangleup B_0) \subseteq  (A_1 \bigtriangleup B_1) \cup ... \cup (A_k \bigtriangleup B_k)$\\
$\Rightarrow (A_1 \cap ... \cap A_k+1) \bigtriangleup (B_1 \cap ... \cap B_k+1) \subseteq (A_1 \bigtriangleup B_1) \cup ... \cup (A_k+1 \bigtriangleup B_k+1) $\\

\subsection*{$\S$ 1.17}
\paragraph*{Условие}
Определить операции $ \cup,  \cap,  \setminus$, через:\\
a)$\bigtriangleup, \cap$
\paragraph*{Доказательство} \mbox{}\\
$\cap = \cap $\\
$ A \cup B = (A \bigtriangleup B) \bigtriangleup ( A \cap B)$\\
$ A \setminus B =  (A \bigtriangleup B) \cap A$
\paragraph*{Условие}
б)$\bigtriangleup, \cup$
\paragraph*{Доказательство} \mbox{}\\
$\cup = \cup $\\
$ A \cap B = ((A \cup B) \bigtriangleup A) \bigtriangleup B $\\
$ A \setminus B =  (A \cup B) \bigtriangleup B$
\paragraph*{Условие}
и)$\setminus, \bigtriangleup$
\paragraph*{Доказательство} \mbox{}\\
$ A \cup B = (A \setminus B) \bigtriangleup $\\
$ A \cap B = (B \setminus (A \setminus B)) $\\
$ \setminus =  \setminus$

\subsection*{$\S$ 1.18}
\paragraph*{Условие}
Доказать, что нельзя определить:\\
a) $\setminus$ через $\cap$ и $\cup $\\
б) $\cup$ через $\cap$ и $\setminus $

\subsection*{$\S$ 1.20}
\paragraph*{Условие}
Найти все подмножества множеств:$\varnothing, \{ \varnothing \}, \{ x \}, \{1, 2\}. $
\paragraph*{Ответ} \mbox{}\\
$\varnothing $ - нет\\
$\{ \varnothing \} - \varnothing$\\ 
$\{ x \} - \varnothing, \{x\}$\\
$\{1, 2\} - \varnothing, \{1\}, \{2\}, \{1, 2\}$

\subsection*{$\S$ 2.1}
\paragraph*{Условие}
Доказать, что существуют A, B и C такие, что:\\
а) $A \times B \neq B \times A$
\paragraph*{Решение} \mbox{}\\
$A = {1}$ и $B = {2}$ 
\paragraph*{Условие}
б) $A \times (B \times С) \neq (A \times B) \times C$
\paragraph*{Решение} \mbox{}\\

\subsection*{$\S$ 2.3}
\paragraph*{Условие}
Доказать, что если A, B, C и D не пусты, то:\\
а) $A \subseteq B$ и $ C \subseteq D $ $\Leftrightarrow A \times C \subseteq B \times D$
б) $A = B$ и $ C = D $ $\Leftrightarrow A \times C = B \times D$
\paragraph*{Решение} \mbox{}\\
Очеивдно доказывается методом от противного.

\subsection*{$\S$ 2.6(а, б, г)}
\paragraph*{Условие}
Доказать, что:
a) $ ( A \cup B ) \times C = ( A \times B ) \cup ( B \times C ) $
б) $ A \times ( B \cup C ) = ( A \times B ) \cup ( A \times C ) $
г) $ ( A \setminus B ) \times C = ( A \times C ) \setminus ( B \times C ) $
\paragraph*{Решение} \mbox{}\\

\section{Отношения и функции}
\subsection*{$\S$ 2.8(а, в)}
\paragraph*{Условие}
\paragraph*{Решение}

\subsection*{$\S$ 2.9(а, в)}
\paragraph*{Условие}
\paragraph*{Решение}

\subsection*{$\S$ 2.12 (б, г)}
\paragraph*{Условие}
\paragraph*{Решение}

\subsection*{$\S$ 2.13}
\paragraph*{Условие}
\paragraph*{Решение}

\subsection*{$\S$ 2.14}
\paragraph*{Условие}
\paragraph*{Решение}

\subsection*{$\S$ 2.22}
\paragraph*{Условие}
\paragraph*{Решение}

\subsection*{$\S$ 2.25(а-д)}
\paragraph*{Условие}
\paragraph*{Решение}

\subsection*{$\S$ 2.31(а)}
\paragraph*{Условие}
\paragraph*{Решение}

\subsection*{$\S$ 2.32(а)}
\paragraph*{Условие}
\paragraph*{Решение}

\subsection*{$\S$ 2.34}
\paragraph*{Условие}
\paragraph*{Решение}

\subsection*{$\S$ 2.35}
\paragraph*{Условие}
\paragraph*{Решение}

\subsection*{$\S$ 2.38(а, в, д)}
\paragraph*{Условие}
\paragraph*{Решение}

\section{Мощности множеств}
\subsection*{$\S$ 4.1}
\paragraph*{Условие}
Доказать, что:\\
$ A \backsim A $ (рефлексивность)\\
Если $ A \backsim B $, то $ B \backsim A $ (симметричность)\\
Если $ A \backsim B $ и $ B \backsim С $, то $ A \backsim С $(транзетивность)\\
\paragraph*{Решение}

\subsection*{$\S$ 4.5}
\paragraph*{Условие}
Доказать, что:\\
а) Всякое подмножество конечного множества конечно\\
б) Объединение конечного числа конечных множест кончено\\
в) Прямое произведение конечного числа конечных множеств конечно\\
\paragraph*{Решение}
Доказательство от противного 

\end{document}
