%%% Поля и разметка страницы %%%
\documentclass[a4paper,12pt]{article}


\textwidth=175mm
\textheight=260mm
\oddsidemargin=-.4mm
\headsep=5mm

\topmargin=-1in
\unitlength=1mm

\usepackage{lscape}		% Для включения альбомных страниц

%%% Кодировки и шрифты %%%
\usepackage{cmap}						% Улучшенный поиск русских слов в полученном pdf-файле
\usepackage[T2A]{fontenc}				% Поддержка русских букв
\usepackage[utf8]{inputenc}				% Кодировка utf8
\usepackage[english, russian]{babel}	% Языки: русский, английский
% \usepackage{pscyr}						% Красивые русские шрифты

%%% Математические пакеты %%%
\usepackage{amsthm,amsfonts,amsmath,amssymb,amscd,mathrsfs} % Математические дополнения от AMS

%%% Оформление абзацев %%%
%\usepackage{indentfirst} % Красная строка

%%% Цвета %%%
\usepackage[usenames]{color}
\usepackage{color}
\usepackage{colortbl}

%%% Таблицы %%%
\usepackage{longtable}					% Длинные таблицы
\usepackage{multirow,makecell,array}	% Улучшенное форматирование таблиц

%%% Общее форматирование
\usepackage[singlelinecheck=off,center]{caption}	% Многострочные подписи
\usepackage{soul}									% Поддержка переносоустойчивых подчёркиваний и зачёркиваний

%%% Библиография %%%
\usepackage{cite} % Красивые ссылки на литературу

%%% Гиперссылки %%%
\usepackage[plainpages=false,pdfpagelabels=false]{hyperref}
\definecolor{linkcolor}{rgb}{0.9,0,0}
\definecolor{citecolor}{rgb}{0,0.6,0}
\definecolor{urlcolor}{rgb}{0,0,1}
\hypersetup{
colorlinks, linkcolor={linkcolor},
citecolor={citecolor}, urlcolor={urlcolor}
}

%%% Изображения %%%
\usepackage{graphicx}		% Подключаем пакет работы с графикой
\graphicspath{{images/}}	% Пути к изображениям

%%% Выравнивание и переносы %%%
\sloppy					% Избавляемся от переполнений
\clubpenalty=10000		% Запрещаем разрыв страницы после первой строки абзаца
\widowpenalty=10000		% Запрещаем разрыв страницы после последней строки абзаца


\usepackage{tikz}

\graphicspath{ {./} }


%%%%%%%%%%%%%%%%%%%%%%%%%%%%%%%%%%%%%%%%%%%%%%%%%%%%%%%%%%%%%%%%%%%%%%%%%%%%%%%%%%%
\begin{document}
\begin{center}
\Huge{Домашнее задание по курсу $"$Математическая логика - 2$"$}
\end{center}

\section{Язык и аксиоматика теории множеств}
\subsection*{$\S$ 1.3}
\paragraph*{Условие}
Доказать, что $\varnothing \not= 	\{ \varnothing \}.$
\paragraph*{Доказательство}
По определению\\
$ x = y \rightleftharpoons \forall t ( t \in x \Leftrightarrow t \in y ).$\\
Пусть $\varnothing = 	\{ \varnothing \} $,
$ \Rightarrow \forall t ( t \in \{ \varnothing \} \Leftrightarrow t \in \varnothing )$
Противоречие для t = $ \varnothing $

\subsection*{$\S$ 1.4}
\paragraph*{Условие}
Доказать, что $\{\{1, 2\}, \{2,3\}\} \not= 	\{ 1,2,3 \}. $
\paragraph*{Доказательство}
По определению\\
$ x = y \rightleftharpoons \forall t ( t \in x \Leftrightarrow t \in y ).$\\
Пусть $\{\{1, 2\}, \{2,3\}\} = 	\{ 1,2,3 \} $,
$ \Rightarrow \forall t ( t \in \{ 1,2,3 \} \Leftrightarrow t \in \{\{1, 2\}, \{2,3\}\} )$
Противоречие для t = $ 1 $

\subsection*{$\S$ 1.6}
\paragraph*{Условие}
Доказать, что $\exists$ лишь одно множество, не имеющее элементов.
\paragraph*{Доказательство}
Пусть $\exists$ два множества X и $X_0$, не имеющих элементов и такие, что $X \not= X_0$\\
$ \Rightarrow \exists t ( t \in X \Rightarrow  t \not\in X_0 )$\\
Противоречие так как $\nexists t \in$ X.

\subsection*{$\S$ 1.8}
\paragraph*{Условие}
Доказать, что множество всех корней многочлена $\alpha (x)=\beta (x) \gamma (x)$ есть объединение множеств корней $\beta (x)$ и $\gamma(x)$.
\paragraph*{Доказательство}
Чтобы докаказать, что множество корней = объединения множеств, надо доказать, что любой корень является либо корнем $\beta (x)$ либо $\gamma(x)$ и что других корней не существует.\\
1) Пусть существует корень $x_0$, который не является корнем ни $\beta (x)$, ни корнем $\gamma(x)$ \\
$\Rightarrow \alpha (x_0) = 0, \beta (x_0) \not= 0,  \gamma (x_0) \not= 0$. Противоречие
2) Пусть $x_0$ корень $\beta (x)$ или  $\gamma(x)$ , тогда  $\beta (x_0) = 0$ или $\gamma(x_0) = 0$ $\Rightarrow \alpha(x_0) = 0$

\subsection*{$\S$ 1.9}
\paragraph*{Условие}
Доказать, что персечение множеств действительных корней многочленов  $\alpha (x) и \beta (x)$ с действительными коэффицентами  совпадает с множеством всех действительных корней $\gamma(x) =\alpha^2 (x) + \beta^2 (x)  $.
\paragraph*{Доказательство}
Чтобы докаказать, что множество корней = персечение множеств, надо доказать, что любой корень из пересейчения является корнем и что других корней не существует.\\
1)Если $x_0$ корень $\alpha (x) и \beta (x)$ $\Rightarrow$ $\gamma(x_0) = 0$
2)Пусть существует корень  $\gamma(x) x_0$, который не является корнем ни $\alpha (x)$, ни корнем $\beta(x)$ \\
Тогда $\gamma(x_0) = 0$ $\Rightarrow \alpha^2 (x_0) + \beta^2 (x_0) = 0 \Rightarrow  \alpha(x_0) =  0 \& \beta (x_0) = 0$

\subsection*{$\S$ 1.11 (а, г, ж)}
\paragraph*{Условие}
Доказать следующие тождества\\
a)$ A \cup A = A \cap A = A $
\paragraph*{Доказательство}
Распишем по определению\\
$\{ Z \mid (Z \in A \vee Z \in A)\} = \{ Z \in A \cup A \mid Z \in A \wedge Z \in A \} = A $\\
Упростим\\
$\{ Z \mid (Z \in A )\} = \{ Z \in A \cup A \mid Z \in A \} = A $ $\Leftrightarrow$ $ A = \{ Z \in A  \mid Z \in A \} = A $  $\Leftrightarrow$ $A=A=A$
\paragraph*{Условие}
г)$ A \cap ( B \cap C ) = ( A \cap B ) \cap C$
\paragraph*{Доказательство} \mbox{}\\
\def\firstcircle{(0,0) circle (1.5cm)}
\def\secondcircle{(45:2cm) circle (1.5cm)}
\def\thirdcircle{(0:2cm) circle (1.5cm)}
\begin{tikzpicture}
	\begin{scope}[shift={(0cm,6cm)}, fill opacity=0.8]
		\begin{scope}% first circle without the second
			\clip \secondcircle;
      		\fill[red] \thirdcircle;
        \end{scope}
        \begin{scope}% first circle without the second
			 \clip \firstcircle;
      		\clip \secondcircle;
     		 \fill[green] \thirdcircle;
        \end{scope}
        \draw \firstcircle node {$A$};
        \draw \secondcircle node {$B$};
        \draw \thirdcircle node {$C$};
    \end{scope}

	 \begin{scope}[shift={(6cm,6cm)}, fill opacity=0.8]
	 	\begin{scope}% first circle without the second
			\clip \firstcircle;
      		\fill[red] \secondcircle;
        \end{scope}
	 	\begin{scope}% first circle without the second
			 \clip \firstcircle;
      		\clip \secondcircle;
     		 \fill[green] \thirdcircle;
        \end{scope}
        \draw \firstcircle node {$A$};
        \draw \secondcircle node  {$B$};
        \draw \thirdcircle node {$C$};
    \end{scope}
\end{tikzpicture}

\paragraph*{Условие}
ж)$A \cup ( B \cap C ) = ( A \cup B ) \cap ( A \cup C )$
\paragraph*{Доказательство} \mbox{}\\

\begin{tikzpicture}
	\begin{scope}[shift={(0cm,6cm)}, fill opacity=0.5]
		\begin{scope}% first circle without the second
			\clip \secondcircle;
      		\fill[red] \thirdcircle;
        \end{scope}
        \begin{scope}% first circle without the second
     		 \fill[green] \firstcircle;
     		 \clip \secondcircle;
      		\fill[green] \thirdcircle;
        \end{scope}

        \draw \firstcircle node {$A$};
        \draw \secondcircle node {$B$};
        \draw \thirdcircle node {$C$};
    \end{scope}

	 \begin{scope}[shift={(6cm,6cm)}, fill opacity=0.5]
	 	\begin{scope}% first circle without the second
	 		\fill[red] \firstcircle;
	 		\fill[red] \secondcircle;
        \end{scope}
        \begin{scope}% first circle without the second
			\fill[yellow] \firstcircle;
      		\fill[yellow] \thirdcircle;
        \end{scope}
	 	\begin{scope}% first circle without the second
        \end{scope}
        \draw \firstcircle node {$A$};
        \draw \secondcircle node  {$B$};
        \draw \thirdcircle node {$C$};
    \end{scope}
\end{tikzpicture}

\subsection*{$\S$ 1.12(в, д, ж, п, т)}

\subsection*{$\S$ 1.13(а, д, к)}

\subsection*{$\S$ 1.14(в, к)}

\subsection*{$\S$ 1.15}
\paragraph*{Условие}
Доказать, что\\
a) $(A_1 \cup ... \cup A_n) \bigtriangleup (B_1 \cup ... \cup B_n) \subseteq (A_1 \bigtriangleup B_1) \cup ... \cup (A_n \bigtriangleup B_n) $
\paragraph*{Доказательство}
Докажем по индукции:\\
\textbf{База индукции}\\
 n=1)  $(A_1) \bigtriangleup (B_1) \subseteq (A_1 \bigtriangleup B_1) $ (очевидно)\\
 n=2)  $(A_1 \cup  A_2) \bigtriangleup (B_1 \cup B_2) \subseteq (A_1 \bigtriangleup B_1) \cup (A_2 \bigtriangleup B_2) $(Доказывалось на уроке)\\
 \textbf{Преположение индукции}\\
 Пусть верно для $\forall n < k$\\
 \textbf{Шаг индукции}\\
 Докажем для k+1\\
 $(A_1 \cup ... \cup A_k+1) \bigtriangleup (B_1 \cup ... \cup B_k+1) \subseteq (A_1 \bigtriangleup B_1) \cup ... \cup (A_k+1 \bigtriangleup B_k+1) $\\
 пусть $ A_0 = A_1 \cup ... \cup A_k и B_0 = B_1 \cup ... \cup B_k$\\
 $(A_1 \cup ... \cup A_{k+1}) \bigtriangleup (B_1 \cup ... \cup B_{k+1}) \Leftrightarrow (A_0 \cup A_{k+1}) \bigtriangleup (B_0 \cup B_{k+1}) \subseteq $ \\
$ \subseteq (A_0 \bigtriangleup B_0) \cup (A_k \bigtriangleup B_k)$\\
$(A_0 \bigtriangleup B_0) \subseteq  (A_1 \bigtriangleup B_1) \cup ... \cup (A_k \bigtriangleup B_k)$\\
$\Rightarrow (A_1 \cup ... \cup A_k+1) \bigtriangleup (B_1 \cup ... \cup B_k+1) \subseteq (A_1 \bigtriangleup B_1) \cup ... \cup (A_k+1 \bigtriangleup B_k+1) $

\paragraph*{Условие}
б) $(A_1 \cap ... \cap A_n) \bigtriangleup (B_1 \cap ... \cap B_n) \subseteq (A_1 \bigtriangleup B_1) \cup ... \cup (A_n \bigtriangleup B_n) $
\paragraph*{Доказательство}
Докажем по индукции:\\
\textbf{База индукции}\\
 n=1)  $(A_1) \bigtriangleup (B_1) \subseteq (A_1 \bigtriangleup B_1) $ (очевидно)\\
 n=2)  $(A_1 \cap  A_2) \bigtriangleup (B_1 \cap B_2) \subseteq (A_1 \bigtriangleup B_1) \cup (A_2 \bigtriangleup B_2) $(Доказывалось на уроке)\\
 \textbf{Преположение индукции}\\
 Пусть верно для $\forall n < k$\\
 \textbf{Шаг индукции}\\
 Докажем для k+1\\
 $(A_1 \cap ... \cap A_k+1) \bigtriangleup (B_1 \cap ... \cap B_k+1) \subseteq (A_1 \bigtriangleup B_1) \cup ... \cup (A_k+1 \bigtriangleup B_k+1) $\\
 пусть $ A_0 = A_1 \cap ... \cap A_k и B_0 = B_1 \cap ... \cap B_k$\\
 $(A_1 \cap ... \cap A_{k+1}) \bigtriangleup (B_1 \cap ... \cap B_{k+1}) \Leftrightarrow (A_0 \cap A_{k+1}) \bigtriangleup (B_0 \cap B_{k+1}) \subseteq $ \\
$ \subseteq (A_0 \bigtriangleup B_0) \cup (A_k \bigtriangleup B_k)$\\
$(A_0 \bigtriangleup B_0) \subseteq  (A_1 \bigtriangleup B_1) \cup ... \cup (A_k \bigtriangleup B_k)$\\
$\Rightarrow (A_1 \cap ... \cap A_k+1) \bigtriangleup (B_1 \cap ... \cap B_k+1) \subseteq (A_1 \bigtriangleup B_1) \cup ... \cup (A_k+1 \bigtriangleup B_k+1) $\\

\subsection*{$\S$ 1.17}
\paragraph*{Условие}
Определить операции $ \cup,  \cap,  \setminus$, через:\\
a)$\bigtriangleup, \cap$
\paragraph*{Доказательство} \mbox{}\\
$\cap = \cap $\\
$ A \cup B = (A \bigtriangleup B) \bigtriangleup ( A \cap B)$\\
$ A \setminus B =  (A \bigtriangleup B) \cap A$
\paragraph*{Условие}
б)$\bigtriangleup, \cup$
\paragraph*{Доказательство} \mbox{}\\
$\cup = \cup $\\
$ A \cap B = ((A \cup B) \bigtriangleup A) \bigtriangleup B $\\
$ A \setminus B =  (A \cup B) \bigtriangleup B$
\paragraph*{Условие}
и)$\setminus, \bigtriangleup$
\paragraph*{Доказательство} \mbox{}\\
$ A \cup B = (A \setminus B) \bigtriangleup $\\
$ A \cap B = (B \setminus (A \setminus B)) $\\
$ \setminus =  \setminus$

\subsection*{$\S$ 1.18}
\paragraph*{Условие}
Доказать, что нельзя определить:\\
a) $\setminus$ через $\cap$ и $\cup $\\
б) $\cup$ через $\cap$ и $\setminus $
\paragraph*{Решение} \mbox{}\\
a)Пусть можно, тогда подставим в это определение множества A и A - их симметрическая разница равна $\varnothing$, но объединение пересечение могут дать только само A\\
б)$\cap$ и $\setminus $ размер множества после выполенеия данных операций не больше, чем максимальное из множество. Но объединение двух множеств дает множество размером суммы размеров. Значит размеры будут разные. $\rightarrow$ нельзя определить.

\subsection*{$\S$ 1.20}
\paragraph*{Условие}
Найти все подмножества множеств:$\varnothing, \{ \varnothing \}, \{ x \}, \{1, 2\}. $
\paragraph*{Ответ} \mbox{}\\
$\varnothing $ - нет\\
$\{ \varnothing \} - \varnothing$\\
$\{ x \} - \varnothing, \{x\}$\\
$\{1, 2\} - \varnothing, \{1\}, \{2\}, \{1, 2\}$

\subsection*{$\S$ 2.1}
\paragraph*{Условие}
Доказать, что существуют A, B и C такие, что:\\
а) $A \times B \neq B \times A$
\paragraph*{Решение} \mbox{}\\
$A = \{1\}$ и $B = \{2\}$, так как, пользуясь определением упорядоченной пары: $ (\{1\},\{2\}) = \{\{1\},\{1,2\}\} \ne \{\{2\},\{2,1\}\} = (\{2\},\{1\})$.
\paragraph*{Условие}
б) $A \times (B \times C) \neq (A \times B) \times C$
\paragraph*{Решение} \mbox{}\\

\subsection*{$\S$ 2.3}
\paragraph*{Условие}
Доказать, что если A, B, C и D не пусты, то:\\
а) $A \subseteq B$ и $ C \subseteq D $ $\Leftrightarrow A \times C \subseteq B \times D$
б) $A = B$ и $ C = D $ $\Leftrightarrow A \times C = B \times D$
\paragraph*{Решение} \mbox{}\\
Очевидно доказывается методом от противного.

\subsection*{$\S$ 2.6(а, б, г)}
\paragraph*{Условие}
Доказать, что:\\
a) $ ( A \cup B ) \times C = ( A \times B ) \cup ( B \times C ) $\\
б) $ A \times ( B \cup C ) = ( A \times B ) \cup ( A \times C ) $\\
г) $ ( A \setminus B ) \times C = ( A \times C ) \setminus ( B \times C ) $
\paragraph*{Решение} \mbox{}\\

\section{Отношения и функции}
\subsection*{$\S$ 2.8(а, в)}
\paragraph*{Условие}
Найти $\delta_R$, $\rho_R$, $R^{-1}$, $R\cdot R$, $R\cdot R^{-1}$, $R^{-1}\cdot R$ для следующих отношений:\\
(а) $R=\{(x,y)|x,y\in \mathbb{N}$ и $x$ делит $y$\}; \\
(в) $R=\{(x,y)|x,y\in \mathbb{D}$ и $x+y\leqslant 0\}$.
\paragraph*{Решение}
(а) Это отношение - всюдуопределенное, так как для любого $x$ существует $y=x$, для которого $x$ делит $y$ $\Rightarrow$ $\delta_R=Pr_1(R)=\mathbb{N}$.\par
Аналогично это отношение - всюдузначное. $\Rightarrow$ $\rho_R=Pr_2(R)=\mathbb{N}$.\par
$R^{-1} = \{(x,y)|(y,x)\in R\} = \{(x,y)|x,y \in \mathbb{N}$ и $y$ делит $x\}$.
\begin{gather*}
R\cdot R \rightleftharpoons \{t\in \mathbb{N}\times \mathbb{N} | \exists u=(u_1,u_2) \exists v=(v_1,v_2) (u\in R\  \& \ v\in R \ \& \ pr_1(t)=\\
= pr_1(u) \ \& \ pr_2(u)=pr_1(v) \ \& \ pr_2(v)=pr_2(t))\}\sim\\
\sim \{(x,y)|x,y\in \mathbb{N} \ \& \ \exists u \exists v (u_2\vdots u_1  \& \ v_2\vdots v_1 \ \& \ x=u_1 \ \& \ u_2 = v_1 \ \& \ v_2 = y)\} \sim \\
\sim \{(x,y)|x,y\in \mathbb{N} \ \& \ y\vdots x\} \Rightarrow R\cdot R = R.
\end{gather*}
(так как $v_2=y\vdots v_1=u_2\vdots u_1=x$, значит $x$ должен делить $y$)
\begin{gather*}
R\cdot R^{-1} \rightleftharpoons \{t\in \mathbb{N}\times \mathbb{N} | \exists u=(u_1,u_2) \exists v=(v_1,v_2) (u\in R\  \& \ v\in R^{-1} \ \& \ pr_1(t)=\\
= pr_1(u) \ \& \ pr_2(u)=pr_1(v) \ \& \ pr_2(v)=pr_2(t))\}\sim\\
\sim \{(x,y)|x,y\in \mathbb{N} \ \& \ \exists u \exists v (u_2\vdots u_1  \& \ v_1\vdots v_2 \ \& \ x=u_1 \ \& \ u_2 = v_1 \ \& \ v_2 = y)\} \sim \\
\sim \{(x,y)|x,y\in \mathbb{N}\} \Rightarrow R\cdot R^{-1} = \mathbb{N}\times \mathbb{N}
\end{gather*}
(так как $u_2=v_1\vdots v_2=y$ и $u_2\vdots u_1=x$, то можно взять в качетстве $u_2$ число, делящееся и на $x$, и на $y$, а сами $x$ и $y$ связаны не будут. Значит, нет дополнительных условий на упорядоченную пару $(x,y)$)
\begin{gather*}
R^{-1}\cdot R \rightleftharpoons \{t\in \mathbb{N}\times \mathbb{N} | \exists u=(u_1,u_2) \exists v=(v_1,v_2) (u\in R^{-1}\  \& \ v\in R \ \& \ pr_1(t)=\\
= pr_1(u) \ \& \ pr_2(u)=pr_1(v) \ \& \ pr_2(v)=pr_2(t))\}\sim\\
\sim \{(x,y)|x,y\in \mathbb{N} \ \& \ \exists u \exists v (u_1\vdots u_2  \& \ v_2\vdots v_1 \ \& \ x=u_1 \ \& \ u_2 = v_1 \ \& \ v_2 = y)\} \sim \\
\sim \{(x,y)|x,y\in \mathbb{N}\} \Rightarrow R^{-1}\cdot R = \mathbb{N}\times \mathbb{N}
\end{gather*}
(так как $x=u_1\vdots u_2=v_1$ и $v_2=y\vdots v_1$, то можно взять в качетстве $v_1$ число~$1$, на которое делится и $x$, и $y$, а сами $x$ и $y$ связаны не будут. Значит, нет дополнительных условий на упорядоченную пару $(x,y)$)\\

\bigskip

(в) Это отношение - всюдуопределенное, так как для любого $x$ существует $y=-x$, для которого $x+y\leqslant 0$ $\Rightarrow \delta_R = Pr_1(R) = \mathbb{D}$.\par
Аналогично это отношение - всюдузначное. $\Rightarrow$ $\rho_R=Pr_2(R)=\mathbb{D}$.\par
$R^{-1} = \{(x,y)|(y,x)\in R\} = R$, так как отношение - симметричное.
\begin{gather*}
R\cdot R \rightleftharpoons \{t\in \mathbb{D}\times \mathbb{D} | \exists u=(u_1,u_2) \exists v=(v_1,v_2) (u\in R\  \& \ v\in R \ \& \ pr_1(t)=\\
= pr_1(u) \ \& \ pr_2(u)=pr_1(v) \ \& \ pr_2(v)=pr_2(t))\}\sim\\
\sim \{(x,y)|x,y\in \mathbb{D} \ \& \ \exists u \exists v (u_1+u_2\leqslant 0\ \& \ v_1+v_2\leqslant 0 \ \& \ x=u_1 \ \& \ u_2 = v_1 \ \& \ v_2 = y)\} \sim \\
\sim \{(x,y)|x,y\in \mathbb{D}\} \Rightarrow R\cdot R = \mathbb{D}\times \mathbb{D}.
\end{gather*}
(условие на $x$, $y$: $x+v_1\leqslant 0$ и $v_1+y\leqslant 0$, но всегда можно взять $v_1$ таким, что оба условия будут выполняться)\par
В силу симметричности отношения $R\cdot R^{-1} = R^{-1}\cdot R = R\cdot R = \mathbb{D}\times \mathbb{D}$.


\subsection*{$\S$ 2.9(а, в)}
\paragraph*{Условие}
Доказать, что:\\
(а) $\delta_R = \varnothing \Leftrightarrow R=\varnothing \Leftrightarrow \rho_R=\varnothing$;\\
(в) $\delta_{R_1\cdot R_2} = R_1^{-1} (\rho_{R_1} \cap \delta_{R_2})$.
\paragraph*{Решение}
(а) $\delta_R = \varnothing \Leftrightarrow \forall u \in  \cup \cup R$ $\forall v$ $(u,v)\notin R$ $\Leftrightarrow R=\varnothing$\par
$\rho_R = \varnothing \Leftrightarrow \forall v \in \cup \cup R$ $\forall u$ $(u,v)\notin R$ $\Leftrightarrow R=\varnothing$\par

\bigskip

(в) $x\in \delta_{R_1\cdot R_2} \Leftrightarrow \exists y:$ $(x,y)\in R_1\cdot R_2$
 $\Leftrightarrow \exists y: \exists u \exists v (u\in R_1 \ \& \ v\in R_2 \ \& \ x=u_1 \ \& \ u_2 = v_1 \ \& \ v_2 = y)$ $\Leftrightarrow$ $\exists y \exists z=u_2=v_1: (x=u_1,z)\in~R_1) \ \& \ (z,y=v_2)\in R_2$ $\Leftrightarrow$ $\exists z: (x,z)\in R_1$ $\&$ $z\in \delta_{R_2}$ $\Leftrightarrow$ $\exists z: (z,x)\in R_1^{-1}$ $\&$ $z\in \rho_{R_1}$ $\&$ $z\in \delta_{R_2}$ $\Leftrightarrow$ $x\in R_1^{-1} (\rho_{R_1} \cap \delta_{R_2})$.
 
\subsection*{$\S$ 2.12 (б, г)}
\paragraph*{Условие}
Доказать, что для любых бинарных отношений:\\
б) $(R^{-1})^{-1} = R$; \\
г) $(R_1 \cap R_2)^{-1} = R_1^{-1} \cap R_2^{-1}$.
\paragraph*{Решение}

\subsection*{$\S$ 2.13}
\paragraph*{Условие}
Для каких бинарных отношений $R$ справедливо $R^{-1} = -R$?
\paragraph*{Решение}
Пусть $R \subseteq A\times B$. \\
1) Предположим, что $x\in A \cap B$. Тогда $(x,x) \in R \Leftrightarrow (x,x) \in R^{-1}$. Если $R^{-1} = -R$, то получим, что $(x,x)$ лежит и в отношении, и в его дополнении, чего быть не может.\\
2) Значит, $A \cap B=\varnothing$. По определению $R \subseteq A\times B$, $R^{-1} \subseteq B\times A$. Значит, $-R=R^{-1}=\varnothing$. Получим, что $R=\varnothing$ и $R=A\times B$, что возможно только при $A=B=\varnothing$.

\subsection*{$\S$ 2.14}
\paragraph*{Условие}
Пусть $A$ и $B$ - конечные множества, состоящие из $m$ и $n$ элементов соответственно.\\
а) Сколько существует бинарных отношений между элементами множеств $A$ и $B$?\\
б) Сколько имеется функций из $A$ в $B$?\\
в) Сколько имеется 1-1-функций из $A$ в $B$?\\
г) При каких $m$ и $n$ существует взаимно однозначное соответствие между $A$ и $B$?
\paragraph*{Решение}
а) Столько, сколько подмножеств у множества упорядоченных пар элементов $A$ и $B$. Всего пар $mn$,  бинарных отношений $2^{mn}$. \\
б) Функция по определению - это всюдуопределенное прямое однозначное бинарное отношение, то есть каждый элемент множества $A$ (из $m$ штук) входит в отношение ровно с одним элементом множества $B$ (из $n$ штук). Тогда всего функций $\underbrace{n\cdot \ldots \cdot n}_m = n^m$. \\
в) Функция $f$ называется 1-1-функцией, если $\forall x_1,x_2,y$: $y=f(x_1)$, $y=f(x_2)\Rightarrow x_1=x_2$. Если $n<m$, то не существует ни одной такой функции, так как не для всех элементов множества $A$ найдется элемент $B$, входящий с ним в отношение. Если $n\geqslant m$, то число таких функций равно $n(n-1)(n-2)\cdot \ldots \cdot (n-m+1)$, так как выбор каждой новой "пары" для элемента множества $A$ уменьшает на $1$ количество возможных пар для прочих элементов множества $A$.\\
г) При $m=n$, тогда и только тогда каждый элемент множества $A$ сможет входить в отношение ровно с одним элементом множества $B$ и наоборот.

\subsection*{$\S$ 2.22}
\paragraph*{Условие}
Доказать, что если $f$ есть функция из $A$ в $B$ и $g$ есть функция из $B$ в $C$, то $f\cdot g$ есть функция из $A$ в $C$. (????Имелось в виду $g\cdot f$ - функция из $A$ в $C$????)
\paragraph*{Решение}
$g\cdot f \leftrightharpoons \{t\in Pr_1(f)\times Pr_2(g)| ...\}$. $Pr_1(f)=A$, $Pr_2(g)=C$ $\Rightarrow$ $g \cdot f: A\to C$. 


\subsection*{$\S$ 2.25(а-д)}
\paragraph*{Условие}
Доказать, что можно установить взаимно однозначное соответствие между множествами:\\
а) $A\times B$ и $B\times A$;\\
б) $A\times (B\times C)$ и $(A\times B)\times C$;\\
в) $(A\times B)^C$ и $A^C\times B^C$;\\
г) $(A^B)^C$ и $A^{B\times C}$;\\
д) $A^{B\cup C}$ и $A^B\times A^C$, если $B\cap C = \varnothing$.
\paragraph*{Решение}
а) Предъявим это соответствие: $\forall x\in A$, $y\in B:$ $(x,y)~\in~A~\times~B \leftrightarrow (y,x)\in B\times A$, то есть любой упорядоченной паре из первого векторного произведения соответствует ровно одна из второго и наоборот.\\
б)

\subsection*{$\S$ 2.31(а)}
\paragraph*{Условие}
Доказать, что для любой функции $f$: \\
а) $f(A\cup B) = f(A)\cup f(B)$.
\paragraph*{Решение}
\begin{gather*}
f[A\cup B] = \{y\in Pr_2(f)|\exists x(x\in A\cup B \ \& \ (x,y)\in f)\} \\
\exists x(x\in A\cup B\ \& \ (x,y)\in f) \sim \exists x(x\in A \vee x\in B \ \& \ (x,y)\in f)  \sim \\
\sim \exists x(x\in A \ \& \ (x,y)\in f) \vee (x\in B \ \& \ (x,y)\in f)
\end{gather*}
Значит, $f[A\cup B] = f[A] \cup f[B]$.

\subsection*{$\S$ 2.32(а)}
\paragraph*{Условие}
Доказать, что для любой функции $f$: \\
а) $f(A\cap B) \subseteq f(A) \cap f(B)$.
\paragraph*{Решение}
\begin{gather*}
f[A\cap B] = \{y\in Pr_2(f)|\exists x (x\in A\cap B \ \& \ (x,y)\in f)\} \\
\exists x(x\in A\cap B \ \& \ (x,y)\in f) \sim \exists x(x\in A \ \& \ x\in B \ \& \ (x,y)\in f) \Rightarrow \\
\Rightarrow \exists x(x\in A \ \&\ (x,y)\in f) \ \& \ (x\in B \ \& \ (x,y)\in f)
\end{gather*}
Значит, $f(A\cap B) \subseteq f(A) \cap f(B)$.

\subsection*{$\S$ 2.34}
\paragraph*{Условие}
Доказать, что $f(A)\setminus f(B) \subseteq f(A\setminus B)$ для любой функции $f$. 
\paragraph*{Решение}
\begin{gather*}
f[A]\setminus f[B] = \{y\in Pr_2(f)|\exists x(x\in A\ \& \ (x,y)\in f)\ \& \ \neg \exists x( x\in B \ \& \ (x,y)\in f)\}.\\
\exists x(x\in A\ \& \ (x,y)\in f)\ \& \ \forall x \neg ( x\in B \ \& \ (x,y)\in f) \Rightarrow \\ \Rightarrow \exists x(x\in A\ \& \ (x,y)\in f)\ \& \  ( x\notin B \vee (x,y)\notin f) \sim \\
\sim \exists x (x\in A \ \& \ x\notin B \ \& \ (x,y)\in f).
\end{gather*}
Значит, $f[A]\setminus f[B] \subseteq f[A\setminus B]$.

\subsection*{$\S$ 2.35}
\paragraph*{Условие}
Доказать, что если в предыдущем примере $f$ есть 1-1-функция, то выполняется равенство.
\paragraph*{Решение}
Пусть $f$ является 1-1-функцией, то есть $\forall x_1,x_2,y: y~=~f(x_1), y~=~f(x_2) \Rightarrow x_1=x_2$. Включение в одну сторону доказано в предыдущей задаче. \\
$y\in f(A\setminus B) \Rightarrow \exists ! x\in A\setminus B: y=f(x) \Rightarrow y\in f(A)$. Так как для  элемента $y$ образа существует единственный прообраз, то $\forall z\in B f(z)\ne y$ (потому что элемент $x$ такой, что $f(x)=y$, лежит в $A\setminus B$, значит, не лежит в $B$). $\Rightarrow y\notin f(B)$ $\Rightarrow y\in f(A)\setminus f(B)$ $\Rightarrow$ $f(A\setminus B) \subseteq f(A)\setminus f(B)$.\\
Вместе с результатом предыдущей задачи получаем:\\
 $f(A)\setminus f(B) = f(A\setminus B)$.
 
\subsection*{$\S$ 2.38(а, в, д)}
\paragraph*{Условие}
Доказать следующие тождества для любой функции $f$:\\
а) $f^{-1} (A\cup B) = f^{-1} (A) \cup f^{-1} (B)$; \\
в) $f^{-1} (A\cap B) = f^{-1} (A) \cap f^{-1} (B)$; \\
д) $f^{-1} (A\setminus B) = f^{-1} (A)\setminus f^{-1}(B)$.
\paragraph*{Решение}
а)
\begin{gather*}
f^{-1} \leftrightharpoons \{t\in Pr_2(f)\times Pr_1(f)|f(pr_2(t)) = pr_1(t) \}, \\
f^{-1}[A\cup B] \leftrightharpoons \{v\in Pr_2(f^{-1})|\exists p(p\in A\cup B) \ \& \ (p,v) \in f^{-1}\} \sim \\
\sim \{v\in Pr_1(f)|\exists p(p\in A\cup B) \ \& \ f(v) = p\} \sim \\
\sim \{v\in Pr_1(f)|\exists p(p\in A \vee p\in B) \ \& \ f(v)=p \}\sim \\
\sim \{v\in Pr_1(f)|\exists p(p\in A) \vee \exists p(p\in B): f(v)=p\} \Rightarrow f^{-1}[A\cup B] = f^{-1}[A] \cup f^{-1}[B].
\end{gather*}
\paragraph*{Условие}
в) $f^{-1} (A\cap B) = f^{-1} (A) \cap f^{-1} (B)$; \\
\paragraph*{Решение}
\begin{gather*}
f^{-1}[A\cap B] \leftrightharpoons \{v\in Pr_2(f^{-1})|\exists p(p\in A\cap B) \ \& \ (p,v) \in f^{-1}\}\\
\exists p(p\in A\cap B) \ \& \ ((p,v) \in f^{-1}) \sim \exists p(p\in A \ \& \ p\in B) \ \& \ ((p,v) \in f^{-1}) \sim \\
\sim \exists p(p\in A) \ \& \ ((p,v) \in f^{-1}) \ \& \ (p\in B)  \Rightarrow f^{-1}[A\cap B] = f^{-1}[A] \cap f^{-1}[B]
\end{gather*}

\section{Мощности множеств}
\subsection*{$\S$ 4.1}
\paragraph*{Условие}
Доказать, что:\\
$ A \backsim A $ (рефлексивность)\\
Если $ A \backsim B $, то $ B \backsim A $ (симметричность)\\
Если $ A \backsim B $ и $ B \backsim С $, то $ A \backsim С $(транзетивность)
\paragraph*{Решение}

\subsection*{$\S$ 4.5}
\paragraph*{Условие}
Доказать, что:\\
а) Всякое подмножество конечного множества конечно\\
б) Объединение конечного числа конечных множест кончено\\
в) Прямое произведение конечного числа конечных множеств конечно
\paragraph*{Доказательство}\mbox{}\\
Доказательство от противного

\subsection*{$\S$ 4.8}
\paragraph*{Условие}
Доказать, что множество тогда и только тогда бесконечно, когда оно эквивалентно некоторому своему подмножеству.
\paragraph*{Доказательство}\mbox{}\\
В условие имеется введу, подмножество не равное множетсву, тк иначе есть контрпример.\\
$\{1\}$ эквивалентен $\{1\}$\\
Докажем лемму о том, что счетное множество $ A \sim A \setminus B $, где В конечное множество.\\
A - счетное, значит все его элементы можно пронумеровать. \\
Возьмем множество  $A \setminus B $, его мы тоже можем пронумеровать, сдвигая каждый раз нумерацию.\\
\\
$\Rightarrow ) $ Если множество бесконечно, то в нем есть счетное подмножество $\Rightarrow$ $\exists$ подмножетсво нашего счетного множества, которое ему  $ \sim $\\
$\Leftarrow )$ Если множетсво $ \sim $ свое подмножеству, то оно не может быть конечным, доказывается от противного $\Rightarrow$ оно бесконечно.

\subsection*{$\S$ 4.10 а}
\paragraph*{Условие}
Пусть область определения счетна, доказать, что область значений этой функции конечна или счетна. \paragraph*{Доказательство}\mbox{}\\
Докажем, что она не более чем счетна.\\
Тк область определения счетна, а каждой точки из области оперделения можно поставить в соотвествие значение функции в этой точки $\Rightarrow$ область значений не более чем счетна $\Rightarrow$ область значений этой функции конечна или счетна.\\

\subsection*{$\S$ 4.13}
\paragraph*{Условие}
Доказать, что:\\
a) Если A бескончено и B - конечное или счетное множество, то $ A \cup B \sim A$
\paragraph*{Доказательство}
Рассмотрим 2 варианта A счетно и A не счетно.\\
Докажем от противного, что в каждом из этих случаях $A \cup B$ счетно и $A \cup B$  не счетно соответственно.
\paragraph*{Условие}
б) Если A бескончено и несчетно, B конечное или счетное множество, то $ A \setminus B \sim A$
\paragraph*{Доказательство}
Пусть это не так $\Rightarrow A \setminus B$ - счетно или конечно. Доказываем от противного, что это невозможно.

\subsection*{$\S$ 4.15}
\paragraph*{Условие}
Доказать, что:\\
a) Множество целых чисел счетно
\paragraph*{Доказательство}
пронумеруем
\begin{center}
  \begin{tabular}{ | l | c | c| c| c| c| c| c| c }
    \hline
    1 & 2 & 3  & 4 & 5  & 6 & 7  & 8 & ...\\ \hline
    0 & 1 & -1 & 2 & -2 & 3 & -3 & 4 & ...\\ \hline
  \end{tabular}
\end{center}
\paragraph*{Условие}
б) Множество рациональных чисел счетно
\paragraph*{Доказательство}
пронумеруем\\
\begin{center}
  \includegraphics[scale=0.5]{image008.jpg}
\end{center}
\paragraph*{Условие}
в) Множество рациональных чисел сегмента $\left[ a, b\right] $ счетно при a < b
\paragraph*{Доказательство}
Множество рациональных чисел сегмента $\left[ a, b\right] $ - беконечно. (тк множество плотно)\\
$\Rightarrow$ оно не менее чем счетно. Но по доказанному выше оно не более, чем счетно $\Rightarrow$ счетно.
\paragraph*{Условие}
г) Множество пар $ \left\langle  x, y \right\rangle $, где х и у - рациональные числа, счетно
\paragraph*{Доказательство}
Множество рациональных чисел счетно.\\
Тогда выпишем все рациональный числа сеткой и докажем, что кол-во пар сечтно аналогично доказатульству 4.15 б\\

\subsection*{$\S$ 4.16}
\paragraph*{Условие}
Доказать, что множество всех конечных последовательностей, составленных из элементов некотрого счетного множества, есть счетное множество.
\paragraph*{Доказательство}
Докажем, что множество последовательностей длины n счетно.\\
Используя 4.15 Г мы знаем, что счетно * счетно = счетно \\
$ \Rightarrow $ счетное$^{n} $ = счетное.\\
Кол-во последовательностей конченой длиный счетно $\Rightarrow$ множество всех последедовательностей конечной длинны тоже счетно.


\subsection*{$\S$ 4.18}
\paragraph*{Условие}
Доказать, что множество многочленов от одной переменной с целыми коэффицентами счетно.
\paragraph*{Доказательство}
Многочлен от одной переменно с целыми коэффицентами представляет из себя конечную последовательных целых чисел $\Rightarrow$ сводится к задаче 4.16

\subsection*{$\S$ 4.19}
\paragraph*{Условие}
Доказать счетность множетсва алгебраических чисел, т. е. чисел, являющихся корнями многочленов от одной переменной с целыми коэвицентами.
\paragraph*{Доказательство}
Кол-во корней у многочлена степени n не более, чем n.\\
Тк кол-во многочленов с целыми коеффицентами от одной перменной счетно (по задаче 4.18), то и кол-во корней счетно.\\
Тк можем пронумеровать.

\subsection*{$\S$ 4.20}
\paragraph*{Условие}
Доказать, что любое множество попарно непересекающихся открытых интервалов на действительной прямой не более чем счетно.
\paragraph*{Доказательство}
Кол-во рациональных чисел счетно. А в каждом интервале есть хотя бы одно рациональное число $\Rightarrow$ интервалов не более чем счетное кол-во.

\subsection*{$\S$ 4.23}
\paragraph*{Условие}
Доказать, что множетсво точек разрыва монотонной функции на дейсвтительной оси не более, чем счетно.
\paragraph*{Доказательство}
У монотонной функции каждая точка разрыва соответствует интервалу на оси Y\\
Эти интервалы попарно непересекающиеся $\Rightarrow$ по здадаче 4.20 множесво не более, чем счетно.

\subsection*{$\S$ 4.24}
\paragraph*{Условие}
Доказать, что:\\
a) $\left( 0, 1\right)  \sim \left[  0, 1 \right]  \sim \left(  0, 1 \right]  \sim \left[  0, 1 \right) $\\
\paragraph*{Доказательство}\mbox{}\\
$\left( 0, 1\right)  \sim \left[  0, 1 \right]$\\
$1/2 \leftrightarrow 0$\\
$1/4 \leftrightarrow 1$\\
$1/k^n \leftrightarrow 4/k^n$\\
остальные числа переведем в себя же соответственно\\
$\left(  0, 1 \right]  \sim \left[  0, 1 \right]$\\
$1 \leftrightarrow 1$\\
$1/2 \leftrightarrow 0$\\
$1/k^n \leftrightarrow 2/k^n$\\
остальные числа переведем в себя же соответственно\\
$\left(  0, 1 \right]  \sim \left[  0, 1 \right)$\\
$x \leftrightarrow 1/2 - \mid 1/2 - x \mid$ (симметрично отнасительно 1/2)
\paragraph*{Условие}
б) $\left[ a, b \right]  \sim \left[  c, d \right] $, где $a < b, c < d$\\
\paragraph*{Доказательство}\mbox{}\\
\begin{center}
  \includegraphics[scale=0.7]{image007.jpg}
\end{center}

\paragraph*{Условие}
в) $\left[ a, b \right] \sim \mathbb{D}$
\paragraph*{Доказательство}\mbox{}\\
по пункту а) $\left( 0, 1\right)  \sim \left[  0, 1 \right]$\\ 
\begin{center}
  \includegraphics[scale=0.7]{image007.jpg}
\end{center}

\subsection*{$\S$ 4.30}
\paragraph*{Условие}
Какова мощность иррациональных чисел?
\paragraph*{Доказательство}
1)Множество иррациональных чисел более чем счетно.\\
Доказательство.\\
Пусть оно счетно. Выпившем все числа по порядку.\\
\begin{center}
  \includegraphics[scale=0.7]{image006.png}
\end{center}
Построим теперь число $C=0, b_1 b_2 b_3 b_4 b_5...$\\
Так что $b_i \neq 0, b_i \neq 9, b_i \neq a_{ii}$\\
Получаем, число, которого нет в таблице, но которое является иррациональным.   


\subsection*{$\S$ 4.31}
\paragraph*{Условие}
Доказать существование трансцендентых (неалгебраических) чисел.
\paragraph*{Доказательство}
Докажем от противного.\\
Пусть их нет. Тогда $ \mathbb{R} \sim $ множество алгебраицеских чисел.\\
Но $ \mathbb{R} $ более чем счетно, а множество всех алгебраических чисел счетно\\
$\Rightarrow$ существуют неалгебраические числа.

\subsection*{$\S$ 4.36}
\paragraph*{Условие}
\paragraph*{Решение}

\subsection*{$\S$ 4.38}
\paragraph*{Условие}
\paragraph*{Решение}

\subsection*{$\S$ 4.39}
\paragraph*{Условие}
\paragraph*{Решение}

\subsection*{$\S$ 4.42}
\paragraph*{Условие}
\paragraph*{Решение}

\section{Отношение эквивелентности}
\subsection*{$\S$ 3.6}
\paragraph*{Условие}
Построить бинарное отношение\\
a)рефлексивное, симметричное, не транзитивное\\
\paragraph*{Решение}
a - нормированное пространство\\
$ r \subseteq a \ast a : (x,y) \in r \leftrightarrow \parallel x - y \parallel \leq \delta $\\
реф. : $\forall x \parallel x - x \parallel = 0 \leq s $\\
симм. : $\parallel x - y \parallel = \parallel y -x  \parallel $\\
Транзитивности нет $\parallel x - y \parallel \leq s $ и $ \parallel y - z  \parallel \leq s \Rightarrow \parallel x - z \parallel \leq s $\\
\paragraph*{Условие}
б)рефлексивное, антисимметричное, не транзитивное\\
\paragraph*{Решение}
$ r \subseteq \mathbb{R} \ast \mathbb{R} : (x,y) \in r \leftrightarrow x \leq y \leq x^2 $\\
реф. : $\forall x (x,x) \in r$\\
антисимметрично: $ x \leq y \leq x^2 $ и $ y \leq x \leq y^2 $ $\Rightarrow x=y$\\
не транз. : $ (2,3) \in r, (3,8) \in r , (2,8) \not\in r$\\
\paragraph*{Условие}
в)рефлексивное, не симметричное, транзитивное\\
\paragraph*{Решение}
$ \leq $ на $ \mathbb{R} $ \\
$ x \leq x$ \\
$ x \leq y, y \leq z \rightarrow x \leq z$\\
$ x \leq y \nrightarrow y \leq x$\\
\paragraph*{Условие}
г)антисимметричное, транзитивное, не рефлексивное\\
\paragraph*{Решение}
$ a \in \mathbb{R} $ \\
$ r \subseteq  \mathbb{R} \times \mathbb{R}$ \\
$ r=\{ (a;a) \} $\\

\subsection*{$\S$ 3.7}
\paragraph*{Условие}
a) Построить бинарное отношение, симметричное, транзитивное, но не рефлексивное.
\paragraph*{Решение}
$r\subseteq \mathbb{R} \times \mathbb{R}$\\
$ a \in \mathbb{R}$\\
$ r =\lbrace ( a; a ) \rbrace $\\
\paragraph*{Условие}
б) Доказать, что если R есть транзитивное и симметричное отношение на множестве A и $\delta_R\cup\rho_R = A$, то R есть эквивалентност на A.
\paragraph*{Решение}
тк $\delta_R\cup\rho_R = A$, то $\exists x \in a \exists y:$\\
либо $(x, y) \in R$ либо $(y, x) \in R$ \\
Из симметричности $(x, y) \in R \wedge (y, x) \in R \rightarrow (x, x) \in R$\\
Те R - отношение эквивалентности.

\subsection*{$\S$ 3.8}
\paragraph*{Условие}
Доказать, что если R симметрично и антисимметрично, то оно транзитивно
\paragraph*{Решение}
Симметричность $(x, y) \in R \rightarrow (y, x) \in R$\\
Антисимметричность $(x, y) \in R \wedge (y, x) \in R \rightarrow y = x$\\
Значит в R лежат только пары вида (x, x) $\Rightarrow$ транзитивно

\subsection*{$\S$ 3.9}
\paragraph*{Условие}
Доказать,что отношение R на множестве a является одновременно эквивалентностью и частичным порядком тогда и только тогда, когда $ R = i_a$
\paragraph*{Решение}
Если $ R = i_a$, то очевидно выоплены рефлексивность, симметричность, антисимметричность и транзитивность.\\
Обратно:\\
Рефлексивность $\Rightarrow \forall x \in a (x, x) \in R \Rightarrow i_a \in R$\\
Сииметричность и антисимметричность $ x = y \Rightarrow R \in  i_a $

\subsection*{$\S$ 3.11}
\paragraph*{Условие}
a - множество всех прямых в $\mathbb{R}^2$, являются ли эквивалентностями следующие отношения?
a) параллельность
б) перпендикулярность
\paragraph*{Решение}
a) является $ a \parallel x $\\
$ x \parallel y \rightarrow y \parallel x$\\
$ x \parallel y \wedge  y \parallel z \rightarrow x \parallel z $\\
б) нет $ \neg x \bot x $ 

\subsection*{$\S$ 3.12}
\paragraph*{Условие}
Определим на $\mathbb{R}$ отношение\\
 a r b $\leftrightarrow$ (a - b) $ \in \mathbb{Q}$\\
Доказать, что r - эквивалентность
\paragraph*{Решение}
a - a = 0 $ \in \mathbb{Q}$\\
a - b $ \in \mathbb{Q} \rightarrow (b - a)=-(a - b) \in \mathbb{Q}$\\
$(a - b) \ in \mathbb{Q} (b - c) \ in \mathbb{Q} \rightarrow (a - c) - (a - b) + (b - c) \in \mathbb{Q}$ 

\subsection*{$\S$ 3.17}
\paragraph*{Условие}
Доказать, что сущесвуют взаимоодназначные соотвествия между классом всех разбиений множества а и семествой всех отношений эквивалентности на a
\paragraph*{Доказательство}
разбиение $\{a_i\}_{i\in\mathbb{I}}$ ставит в соответствие $(x, y) \in r \leftrightarrow \exists i: x \in a_i , y \in a_i$

\subsection*{$\S$ 3.19}
\paragraph*{Условие}
\paragraph*{Решение}

\subsection*{$\S$ 3.20}
\paragraph*{Условие}
\paragraph*{Решение}


\section{Упорядоченные множества и ординальные числа}
\subsection*{$\S$ 3.30}
\paragraph*{Условие}
a)Доказать,что всякое частично упорядоченно множество содержит не более одного наибольшего элемента.
б)Доказать, что наибольший (наименьший) элемент частично упорядоченного множества является единственным максимальным(минимальным) элементом.
в)Построить пример частично упорядоченного множества, имеющего точно один минимальный элемент, но не имеющего наименьшего элемента.
\paragraph*{Решение}
а) и б) доказюватся от противного.
в) Возьмем множество всех целых чисел, где нет наименьшего и минимального элемента и добавим к этому множеству элемент а, котороый ни с одним из остальных не сравним, что делает его минимальным, но не наименьшим. Получаем множество с 1 минимальным и без наименьших.
\subsection*{$\S$ 3.39}
\paragraph*{Условие}
Доказать, что любое частично упорядоченное множество A изоморфно некоторой системе подмножеств А, упорядечнной вклчением $ \subseteq $.
\paragraph*{Решение}
Докажем по построению. Построим систему подмножеств начиная с минимальных и наименьших элементов. Построим из них множества из 1ого элемента, дальше, возьмем эелемнты которые сравнимы с ними и прибавим к ним эти элементы. И опять найдем наименьшеи и минимальных, убрав предыдущие.

\subsection*{$\S$ 3.42}
\paragraph*{Условие}
Построить линейный порядок на множестве:\\
a) $ \mathscr{N}^2 $\\
б) $ \mathscr{N} \cup \mathscr{N}^2 \cup ... \cup \mathscr{N}^n \cup $\\
в) $ \mathscr{B}$ компелксных чисел
\paragraph*{Решение}

\subsection*{$\S$ 3.49}
\paragraph*{Условие}
\paragraph*{Решение}

\subsection*{$\S$ 3.54}
\paragraph*{Условие} Доказать, что любое подмножество множества P(A), частично упорядоченное по включению, имеет точную верхнюю грань и точную нижнюю грань.
\paragraph*{Решение}

\subsection*{$\S$ 5.13}
\paragraph*{Условие}
\paragraph*{Решение}

\subsection*{$\S$ 5.14}
\paragraph*{Условие}
\paragraph*{Решение}

\subsection*{$\S$ 5.38}
\paragraph*{Условие}
\paragraph*{Решение}

\subsection*{$\S$ 5.46}
\paragraph*{Условие}
\paragraph*{Решение}

\subsection*{$\S$ 5.47}
\paragraph*{Условие}
\paragraph*{Решение}

\subsection*{$\S$ 5.48}
\paragraph*{Условие}
\paragraph*{Решение}

\subsection*{$\S$ 5.50}
\paragraph*{Условие}
\paragraph*{Решение}

\subsection*{$\S$ 5.51}
\paragraph*{Условие}
\paragraph*{Решение}

\subsection*{$\S$ 5.66}
\paragraph*{Условие}
\paragraph*{Решение}

\end{document}
